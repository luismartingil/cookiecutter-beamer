

\documentclass[compress,xcolor={table}]{beamer}

\setbeamersize{text margin left = 2.5em}

\usepackage{tipa}
\usepackage{tikz}
\usepackage{lmodern}
\usepackage{epstopdf}
\usepackage{listings}
\usepackage[abs]{overpic}
\usepackage{minted}

\definecolor{dkgreen}{rgb}{0,0.6,0}
\definecolor{gray}{rgb}{0.5,0.5,0.5}
\definecolor{mauve}{rgb}{0.58,0,0.82}
\definecolor{gray}{rgb}{0.4,0.4,0.4}
\definecolor{darkblue}{rgb}{0.0,0.0,0.6}
\definecolor{cyan}{rgb}{0.0,0.6,0.6}
\definecolor{my_color}{RGB}{114, 129, 210}
\definecolor{mycompany_red}{RGB}{217,31,20}
\definecolor{mycompany_blue}{RGB}{4,118,176}

\newenvironment{changemargin}[2]{% 
  \begin{list}{}{% 
      \setlength{\topsep}{0pt}% 
      \setlength{\leftmargin}{#1}% 
      \setlength{\rightmargin}{#2}% 
      \setlength{\listparindent}{\parindent}% 
      \setlength{\itemindent}{\parindent}% 
      \setlength{\parsep}{\parskip}% 
    }% 
  \item[]}
{\end{list}} 
 
\lstset{ %
  basicstyle=\tiny,
  framexleftmargin=3mm,
  xleftmargin=3mm,
  numbers=left,
  numberstyle=\tiny\color{gray},
  stepnumber=1,
  numbersep=2pt,
  showspaces=false,
  showstringspaces=false,
  showtabs=false,
  frame=single,
  rulecolor=\color{gray},
  tabsize=1,
  breaklines=true,
  breakatwhitespace=true,
  escapeinside={\%*}{*)},
  morekeywords={*,...}
}

\lstdefinelanguage{XML}
{
basicstyle=\ttfamily\color{darkblue}\bfseries,
morestring=[b]'',
morestring=[s]{>}{<},
morecomment=[s]{<?}{?>},
stringstyle=\color{black},
identifierstyle=\color{darkblue},
keywordstyle=\color{cyan},
morekeywords={xmlns,version,type}
}


\usetikzlibrary{positioning}
\usetikzlibrary{arrows}
\usetikzlibrary{arrows,decorations.pathmorphing,through,backgrounds,positioning,fit,petri}
\usetikzlibrary{shapes,shadows}
\usetikzlibrary{shapes.multipart}
\usetikzlibrary{calc}
\usetikzlibrary{decorations.pathreplacing}
\usepackage{animate}

\setbeamertemplate{navigation symbols}{}
\pgfdeclareimage[height=0.45cm]{logos/mycompany-logo.png}{logos/mycompany-logo.png}
\logo{\pgfuseimage{logos/mycompany-logo.png}}
\setbeamertemplate{items}[ball] 
\setbeamertemplate{blocks}[rounded][shadow=true] 
\usetheme{Warsaw} 

\setbeamercolor{title}{fg=white}
\setbeamercolor{frametitle}{fg=white}
\setbeamercolor{structure}{fg=my_color}
\setbeamercolor*{palette quaternary}{fg=white,bg=UniBlue!30!black}

%\useoutertheme{infolines}
%\useoutertheme[subsection=true]{tree}
%\useoutertheme[subsection=true]{smoothtree}
%\useoutertheme[subsection=true]{tree}
\useoutertheme{miniframes}
\useinnertheme{rectangles}

% fix clipped headline (bug with Warsaw/infolines)
%\addtobeamertemplate{headline}{}{\vskip2pt}

\beamersetuncovermixins{\opaqueness<1>{25}}{\opaqueness<2->{15}}

\begin{document}



\title{ {{cookiecutter.project_name}} }
\author{ {{cookiecutter.email}} }
\institute{ {{cookiecutter.company}} \textsuperscript{\textregistered}}

\date{\today}
\defbeamertemplate*{footline}{shadow theme}
{%
  \leavevmode%
  \hbox{\begin{beamercolorbox}[wd=.5\paperwidth,ht=2.5ex,dp=1.125ex,leftskip=.3cm plus1fil,rightskip=.3cm]{author in head/foot}%
    \usebeamerfont{author in head/foot}\insertframenumber\,/\,\inserttotalframenumber\hfill\insertshortauthor
  \end{beamercolorbox}%
  \begin{beamercolorbox}[wd=.5\paperwidth,ht=2.5ex,dp=1.125ex,leftskip=.3cm,rightskip=.3cm plus1fil]{title in head/foot}%
    \usebeamerfont{title in head/foot}\insertshorttitle%
  \end{beamercolorbox}}%
  \vskip0pt%
}

\usebackgroundtemplate{\centering\includegraphics[width=\paperwidth]{logos/mycompany-logo-watermark.png}} 

\begin{frame}
  \titlepage 
\end{frame}

\begin{frame}{\contentsname}
\begin{columns}[onlytextwidth]
\begin{column}{.48\textwidth}
\scriptsize\tableofcontents[sections=1-1]
\end{column}
\begin{column}{.48\textwidth}
\scriptsize\tableofcontents[sections=2-2]
\end{column}
\end{columns}
\end{frame}

\begin{frame}
\frametitle{Table of contents}
\scriptsize\tableofcontents
\end{frame}

\AtBeginSection[]
{
  \begin{frame}<beamer>
    \frametitle{Contents}
    \scriptsize\tableofcontents[currentsection,hideothersubsections]
  \end{frame}
}

% sections
\section{Section1}
%
\subsection{Subsection a}
\begin{frame}[label=sectiona]
\frametitle{subsection a title} 

{{cookiecutter.project_short_description}}

\begin{enumerate}
\item Bla bla bla.
\item Bla bla bla.
\item Bla bla bla.
\item Bla bla bla.
\end{enumerate}
\end{frame}


%
\subsection{Subsection b}
\begin{frame}[label=sectionb]
\frametitle{subsection b title} 
\begin{enumerate}
\item Bla bla bla.
\item Bla bla bla.
\item Bla bla bla.
\item Bla bla bla.
\end{enumerate}
\end{frame}

\section{Section2}
%
\subsection{Subsection c}
\begin{frame}[label=sectionc, t, fragile]
\frametitle{subsection c title} 

Highlighted source code using minted.

\usemintedstyle{perldoc}

\begin{minted}[linenos,
             numbersep=12pt,
             firstnumber=1,
             firstline=1,
             frame=lines,
             framerule=0.1pt,
             rulecolor=\definecolor{bg}{rgb}{0.95,0.95,0.95},
             fontsize=\tiny,
             framesep=2mm]{python}
import numpy as np

def incmatrix(genl1,genl2):
    m = len(genl1)
    n = len(genl2)
    M = None #to become the incidence matrix
    VT = np.zeros((n*m,1), int)  #dummy variable

    #compute the bitwise xor matrix
    M1 = bitxormatrix(genl1)
    M2 = np.triu(bitxormatrix(genl2),1)

    # Removed lines

    return M
\end{minted}

\end{frame}


%
\subsection{Subsection d}
\begin{frame}[label=sectiond]
\frametitle{subsection d title} 
\begin{itemize}
\item Bla bla bla.
\item Bla bla bla.
\item Bla bla bla.
\item Bla bla bla.
\end{itemize}
\end{frame}


\end{document}


